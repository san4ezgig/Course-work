\documentclass[a4paper,14pt]{extarticle}
\usepackage{graphicx}
\usepackage[12pt]{extsizes}
\graphicspath{{pictures/}}
\DeclareGraphicsExtensions{.pdf,.png,.jpg}
\usepackage[utf8]{inputenc}
\usepackage[english,russian]{babel}
\usepackage{amssymb}
\usepackage{amsmath}
\usepackage{xcolor}
\renewcommand
\linespread{1}
\usepackage[left=3cm,right=1cm, top=2cm, bottom=2cm, bindingoffset=0cm]{geometry}
\usepackage{xcolor}

\begin{document}
	\begin{titlepage}
		\begin{center}
			\fontsize{13pt}{13pt}\selectfont
			\textbf{МИНИСТЕРСТВО ОБРАЗОВАНИЯ РЕСПУБЛИКИ БЕЛАРУСЬ}\\
			\vspace{0.5cm}
			\textbf{БЕЛОРУССКИЙ ГОСУДАРСТВЕННЫЙ УНИВЕРСИТЕТ}\\
			\vspace{0.5cm}
			\textbf{ФАКУЛЬТЕТ ПРИКЛАДНОЙ МАТЕМАТИКИ И ИНФОРМАТИКИ}\\
			\vspace{0.5cm}
			\fontsize{12pt}{12pt}\selectfont
			\textbf{Кaфедрa теории вероятностей и мaтемaтической стaтистики}\\
			\vspace{3.0cm}
			\fontsize{18pt}{18pt}\selectfont
			\vspace{0.5cm}
			Аннотация к дипломной работе\\
			\vspace{0.5cm}
			\textbf{Исследование системы MAP/G/1 работающей с использованием энергии, генерируемой в режиме реального времени}\\
			\vspace{1.5cm}
			\fontsize{16pt}{16pt}\selectfont
			
		\end{center}

		\fontsize{14pt}{14pt}\selectfont
		\hspace{-0.25cm}
		\def\arraystretch{1.2}
		\begin{center}
			Джига Александр Олегович \\
			\vspace{4cm}
			Научный руководитель - профессор Дудин Александр Николаевич
			\vspace{3cm}
		\end{center}
		\vspace{1.5cm}
		\begin{center}
			\fontsize{16pt}{16pt}\selectfont
			Минск, 2019
		\end{center}
	\end{titlepage}
\setcounter{page}{2}
\begin{center}
	\section*{Реферат}
	\begin{flushleft}
		$\qquad$Ключевые слова: СИСТЕМА МАССОВОГО ОБСЛУЖИВАНИЯ,
		СИСТЕМА MAP|G|1, ПЕРЕХОДНЫЕ ВЕРОЯТНОСТИ, КРИТЕРИЙ
		ЭРГОДИЧНОСТИ, СТАЦИОНАРНОЕ РАСПРЕДЕНИЯ ВЕРОЯТНОСТЕЙ.\\
		$\qquad$Объектом исследования является система массового обслуживания типа
		$MAP|G|1$ с генерацией энергии. Цель работы – изучить систему, описать модель
		и исследовать ее поведение в различных случаях. Найдены переходные
		вероятности системы. Построена матрица переходных вероятностей. Найдено
		условие эргодичности.
	\end{flushleft}
\newpage
\section*{Abstract}
	\begin{flushleft}
		$\qquad$ Key words: QUEUEING SYSTEM, MAP | G | 1 SYSTEM, TRANSITION
		PROBABILITIES, ERGODICITY CRITERION, STATIONARY DISTRIBUTION
		OF PROBABILITIES.\\
		$\qquad$The object of the study is a queuing system of the $MAP|G|1$ type with the
		generation of energy. The aim of the work is to study the system, describe the
		model and investigate its behavior in various cases. The transition probabilities
		of the system are found. A matrix of transition probabilities is constructed. The
		ergodicity condition is found.
		parameters.

	\end{flushleft}
\end{center}
\end{document}